%%%%%%%%%%%%%%%%%%%%%%%%%%%%%%%%%%%%%%%%%%%%%%%%%%%%
% This will help you in writing your homebook
% Remember that the character % is a comment in latex
%
% chapter 5
\chapter{Testbench}
\label{cha5}

\section{Memory component}

For the testbench two more components have been written to mimic ricv memories. Both of them
work on the negative edge of the clock and only the needed cells has been allocated in order 
to speed up the simulation.

For data memory has been used the data\_memory entity. It is a memory with one single address used
for writing and reading, one input port for data to be written and one output port 
for data to be reed. Two different signal activate the writing or the reading on the memory.

For instruction memory has been used the instruction\_memory entity. It has one input port
for address and one output port for read instruction.

\section{ASM code}

For testbench an assembly code has been written that find the minimum absolute value in a 
set of data, with cardinality six. Through a simulator it has been tested and the binary instructions code has been 
stored in instruction memory while the data in data memory, the two previous entity, both in the right address.

The following set of instructions is used to calculate the absolute value.
\begin{lstlisting}
    lw x8,0(x4)       
    srai x9,x8,31     
    xor x10,x8,x9     
    andi x9,x9,0x1    
    add x10,x10,x9   
\end{lstlisting}

If the simulation discovers a right behavior at the end, while the last instruction performed an infinite loop 
on itself, will be possible to see in the memory, at position seven, the value 3 that correspond to the minimum 
absolute value.

\section{ABS function}

The abs operation that has been created it's not a real instruction, it can't be compiled and doesn't exist
an absolute instruction for integer registers. In the previous version of asm code the absolute value was calculated 
through different operations while now they are now substituted in the following way.

\begin{lstlisting}
    lw x8,0(x4)      
    abs x10,x8,0
\end{lstlisting}

For this reason, after removing the four instruction that calculated
the absolute value of a register, during ASM simulation a NOP instruction has been added to calculate the
offset values used from the jumps. While, in the instruction memory, the NOP instruction has been substituted from 
the abs instruction, codified manually, bit by bit, to achieve the right behavior.
At the end the result must be the same as in the previous version.


\section{Conclusion}

The first simulation lest 101 clock cycles the second one least only 83.