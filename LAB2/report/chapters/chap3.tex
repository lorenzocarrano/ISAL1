%%%%%%%%%%%%%%%%%%%%%%%%%%%%%%%%%%%%%%%%%%%%%%%%%%%%
% This will help you in writing your homebook
% Remember that the character % is a comment in latex
%
% chapter 3
\chapter{MBE Multiplier}
\label{cha3}
\section{Introduction}
The goal of this section was to design a Modified Booth Encoder multiplier, a radix-4 variant of the standard operation, to be utilized in the provided floating point multiplier.
Even though the operation was executed in 32 bits, the last 8 bits of the multiplicands were set to zero. It was then sufficient to implement a 24 bit version of the MBE.   
With this algorithm, the partial products are obtained by dividing the multiplier in 3-bit slices, where each slice shares 1 bit with the consecutive one; these are
used to retrieve the corresponding partial product from the following table.

\begin{center}
\begin{tabularx}{0.5\textwidth} { 
    | >{\centering\arraybackslash}X 
    | >{\centering\arraybackslash}X| }
   \hline
   $b_{2i+1} b_{2i} b_{2i-1}$ & $p_{i}$ \\
   \hline
   000  & 0 \\
   001  & a \\
   010  & a \\
   011  & 2a \\
   100  & -2a \\
   101  & -a \\
   110  & -a \\
   111  & 0 \\
  \hline
\end{tabularx}
\end{center}

The sum of the partial products will be the result of the multiplication.

%image

\section{Partial Products Generation}
In order to obtain all the needed partial products, a simple 4-input multiplexer was implemented. It receives as inputs 0, a and 2a, and as selection bits the result of this
operation: $ (b_{2i} b_{2i-1}) xor (b_{2i+1} b_{2i+1})$ as shown in the following image. The $b_{2i+1}$ bit is also used as the sign of the partial product: if the result is negative,
the output of the multiplexer is negated in order to compute its 2's complement. 
% image

\section{Sign Extension}
The sign extension for unsigned multiplication is a technique that allows to reduce the number of redundant additions when computing the sum of the partial products.
In particular, when an operand is negative, the string of $'1'$ in the MSBs causes an unnecessary number of bits to be included in the final sum. The algorithm avoids this problem,
and also takes care of the $+1$ needed to compute the 2's complement without the need of an additional operation. The partial products are modified as shown in the following image,
where S is the sign of the operand. 
% image

\section{Dadda-Tree}




