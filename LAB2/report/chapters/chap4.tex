%%%%%%%%%%%%%%%%%%%%%%%%%%%%%%%%%%%%%%%%%%%%%%%%%%%%
% This will help you in writing your homebook
% Remember that the character % is a comment in latex
%
% chapter 4
\chapter{Comparison Between Architectures}
\label{cha4}

\begin{center}
\begin{tabularx}{0.5\textwidth} { 
    | >{\centering\arraybackslash}X
    | >{\centering\arraybackslash}X 
    | >{\centering\arraybackslash}X| }
   \hline
    & \textbf{Frequency} & \textbf{Area} \\ 
   \hline
   Base Architecture  & 625.00MHz & 4104.1 \\
   \hline
   CSA Multiplier & 223.21 MHz & 4906.9\\
   \hline
   PPARCH Multiplier & 641.03MHz & 4182.8\\
   \hline
   Optimized with registers  & 1.08GHz & 4656.9 \\
   \hline
   Compiled with compile ultra  & 714.29 MHz & 4478.9 \\
   \hline
   MBE Standard Multiplier  & 357.14 MHz & 4824.2 \\
   \hline
   MBE opt. with registers  & 1.12 GHz & 6318.3 \\ 
   \hline
   MBE compiled with compile ultra  & 653.59MHz & 4661.6 \\
  \hline
\end{tabularx}
\end{center}

In general, it can be noticed that the best performance is obtained by means of optimization using registers.
Compiling with compile ultra command returns a solution that is a good compromise betweenn performance and total area.
For example, compiling the base architecture using compile ultra, it can be achieved a performance growth of, 14\% at the cost of 9\% of additional area, while
modifying the base architecture inserting registers, a growth of 73\% of performance is achieved, at the cost of about 13\% of additional area.
Thus, in general, if  the area constraints are not so critical, optimizing by means of registers can be strongly efficient in terms of performance.