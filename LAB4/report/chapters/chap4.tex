%%%%%%%%%%%%%%%%%%%%%%%%%%%%%%%%%%%%%%%%%%%%%%%%%%%%
% This will help you in writing your homebook
% Remember that the character % is a comment in latex
%
% chapter 4
\chapter{Floating Point Multiplier}
\label{cha4}

\section{Testbench Modifications}

The floating point multiplier, in which the MBE multiplier tested in the previous section is used for the multiplication of the significands, is a pipelined architecture. 
This means that the results of the multiplications are ready some clock cycles after the generation of the inputs. In order to compensate for the added latency, 
the verification has to be delayed with respect to the generation of the inputs.
After an initial latency of six clock cycles necessary to perform the first multiplication, the testbench handles the multiplier in a pipelined way, 
performing and verifying one operation every three clock cycles. This is possible by saving the generated inputs in a chain of two registers; since the FSM is composed
of three states, this is equivalent to delaying the inputs of six clock cycles. When a new result is ready, it is confronted with the correct one computed using the inputs 
present in the second register of the chain.
The FSM is composed of three states:

\begin{itemize}
    \item Initial: all control signals are initialized, then the FSM moves on to the WAIT state.
    \item Wait: the inputs are saved in order to be delayed, and the results of the multiplications are displayed. Two flags are set, in\_inter.ready = 0 
    to stop the generation of new inputs, and out\_inter.valid = 1 to signal the presence of a new valid output to confront with the reference one.
    \item Send: when the results have been compared, in\_inter.ready is set to 1 to generate a new input and out\_inter.valid = 0.
\end{itemize}

\section{Input Generation}

Since it is not possible to generate random single-precision floating point numbers, random 32-bits sequences are generated. Using the functions \$shortrealtobits() 
and \$bitstoshortreal() it is possible to convert from floating point numbers to their binary representation and viceversa. To make the results more readable, 
some constraints have been applied to the input generation: in particular, the bits from 30 to 23, which represent the exponent of the floating point number, 
are forced to be equal to 128 for both generated inputs.

\section{Testbench Run}

During the testbench run no mismatches occurred, as shown in the transcript file.\\
\\
\# FPMUL: input A = 3.210777, input B = -2.400374\\
\# FPMUL: input A = 01000000010011010111110101011111, input B = 11000000000110011001111110111001\\
\# FPMUL: output OUT = -7.707065\\
\# FPMUL: output OUT = 11000000111101101010000001000110\\
\# refmod: input A = 3.210777, input B = -2.400374, output OUT = -7.707065\\
\# refmod: input A = 01000000010011010111110101011111, input B = 01000000010011010111110101011111, output OUT = 11000000111101101010000001000110\\
\# UVM\_INFO @ 3015: uvm\_test\_top.env\_h.comp [Comparator Match] \\
\# FPMUL: input A = 3.084183, input B = -3.686763\\
\# FPMUL: input A = 01000000010001010110001100111111, input B = 11000000011010111111001111101110\\
\# FPMUL: output OUT = -11.370651\\
\# FPMUL: output OUT = 11000001001101011110111000110000\\
\# refmod: input A = 3.084183, input B = -3.686763, output OUT = -11.370651\\
\# refmod: input A = 01000000010001010110001100111111, input B = 01000000010001010110001100111111, output OUT = 11000001001101011110111000110000\\
\# UVM\_INFO @ 3045: uvm\_test\_top.env\_h.comp [Comparator Match] \\
\# UVM\_INFO verilog\_src/uvm-1.1d/src/base/uvm\_objection.svh(1267) @ 3045: reporter [TEST\_DONE] 'run' phase is ready to proceed to the 'extract' phase\\
\# UVM\_INFO ../tb/env.sv(42) @ 3045: uvm\_test\_top.env\_h [env] Reporting matched 101\\
\# \\
\# --- UVM Report Summary ---\\
\# \\
\# ** Report counts by severity\\
\# UVM\_INFO :  106\\
\# UVM\_WARNING :    0\\
\# UVM\_ERROR :    0\\
\# UVM\_FATAL :    0\\
\# ** Report counts by id\\
\# [Comparator Match]   101\\
\# [Questa UVM]     2\\
\# [RNTST]     1\\
\# [TEST\_DONE]     1\\
\# [env]     1\\
