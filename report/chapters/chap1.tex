%%%%%%%%%%%%%%%%%%%%%%%%%%%%%%%%%%%%%%%%%%%%%%%%%%%%
% This will help you in writing your homebook
% Remember that the character % is a comment in latex
%
% chapter 1
\chapter{Reference model development}
\label{chap1}

%%%%%%%%%%%%%%%%%%%%%%%%%%%%%%%%%%%%%%%%%%%%%%%%%%%%%%%%%%%
% you can organize a chapter using sections -> \section{Simulating an inverter}
% or subsections -> \subsection{simulating a particular type of inverter}

%%%%%%   First section
\section{Introduction}

The goal of this laboratory is to design a Finite Impulse Filter filter (FIR) with a cut frequency of 2 kHz.
Filter has is design according two parameter: order and number of bits. The order employ for the following filter
is 10 and the number of bits are 9.

Before starting with filter design it's needed to develop a prototype version that ensure the final 
result of request implementation.

\section{Design the filter with Matlab/Octave}

First step is the generation of coefficients. To do this Matlab function fir1 has been used.
The coefficients are shown in table \ref{tab:1}. % here is the reference to the table below

\begin{table}[ht]
\centering
\begin{tabular}{c|c|c}
\toprule
Number & Quantize & Normalize \\
\midrule
0 & -1 & 1 \\
1 & -2 & 1 \\
2 & -4 & 1 \\
3 & 8 & 0 \\
4 & 35 & 1 \\
5 & 50 & 1 \\
6 & 35 & 1 \\
7 & 8 & 1 \\
8 & -4 & 1 \\
9 & -2 & 1 \\
10 & -1 & 1 \\
\bottomrule
\end{tabular}
\caption{All coefficients.}
\label{tab:1}
\end{table}

Always staying in Matlab and using the previous coefficients, a further Matlab script is executes in order to perform
different simulation with prototype filter with a cut-off frequency of 2 kHz and a sampling frequency of 10 kHz. The input signal
used is an average between two sinusoidal waves respectively at 500 HZ and 4.5 kHz.
After this execution two files have been generated:

\begin{enumerate}
	\item \emph{sample.txt}, which contains the sample values that have fed the input of FIR;
	\item \emph{result.txt}, which contains the output values that has been elaborated from our FIR.
\end{enumerate}

\section{C prototype}

A C program language has been written to have a 
fixed point implementation of FIR that use the following formula:

\begin{displaymath}
y_i = \sum_{n=0}^{10}{x_{i-m} \cdot a_n}
\end{displaymath}

Thanks this program is possible to evaluate the performance of fixed version respect to 
Matlab execution.

\subsection{Evaluate the THD}

The purpose of this script is to evaluate the Total Harmonic Distortion (THD) trying to react a value that is
maximum -30dB. If an amount of tollerance is avaiable maybe will be possible to reduce 
the bit numbers and so to reduce the size of FIR design.

%% add value of THD caculated each time

In first hand, with 9 bits used for data, the THD is -40 dB. Trying to reduce the number of bits to 8 the value of THD obtain 
is still acceptable, it's -33 dB. When a further reduction has been applied the value of THD a 
not allowed value is returned, with 7 bits THD is -27dB.

In the end, for the final implementation of FIR 8 bits have been used in order to achieve the THD request and 
to reduce the area.

%% vedi i ths a 9, 8 e 7 bits