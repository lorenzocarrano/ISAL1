%%%%%%%%%%%%%%%%%%%%%%%%%%%%%%%%%%%%%%%%%%%%%%%%%%%%
% This will help you in writing your homebook
% Remember that the character % is a comment in latex
%
% chapter 3
\chapter{Advanced architecture development}
\label{cha3}

In this section the purpose is to improve the FIR performance. In first and unfolding 
technique has been used to improve the throughput and then pipeline technique has 
been used to reduce the longest path and improve maximum frequency.

\section{Unfolding}

Unfolding of order 3 has been applied to FIR filter (N = 3) and the equations derived
to build the new system are the following:

\begin{equation}
\begin{split}
    y[3n] = a_0 \cdot x[3n] + a_1 \cdot x[3(n-1) + 2] + a_2 \cdot x[3(n-1) + 1] + \\
    a_3 \cdot x[3(n-1)] + a_4 \cdot x[3(n-2) + 2] + a_5 \cdot x[3(n-2) + 1] + a_6 \cdot x[3(n-2)] +  \\
    a_7 \cdot x[3(n-3) + 2] + a_8 \cdot x[3(n-3) + 1] + a_9 \cdot x[3(n-3)] + a_{10} \cdot x[3(n-4) + 2]
\end{split}
\end{equation}

\begin{equation}
\begin{split}
    y[3n + 1] = a_0 \cdot x[3n + 1] + a_1 \cdot x[3n] + a_2 \cdot x[3(n-1) + 2] + \\
    a_3 \cdot x[3(n-1) + 1] + a_4 \cdot x[3(n-1)] + a_5 \cdot x[3(n-2) + 2] + a_6 \cdot x[3(n-2)+1] +  \\
    a_7 \cdot x[3(n-2)] + a_8 \cdot x[3(n-3)+2] + a_9 \cdot x[3(n-3) + 1] + a_{10} \cdot x[3(n-3)]
\end{split}
\end{equation}

\begin{equation}
\begin{split}
    y[3n + 2] = a_0 \cdot x[3n + 2] + a_1 \cdot x[3n + 1] + a_2 \cdot x[3n] + a_3 \cdot x[3(n-1) + 2] +\\
    a_4 \cdot x[3(n-1) + 1] + a_5 \cdot x[3(n-1)] + a_6 \cdot x[3(n-2) + 2] +  a_7 \cdot x[3(n-3) + 1] +\\
    a_8 \cdot x[3(n-2)] + a_9 \cdot x[3(n-3) + 2] + a_[10] \cdot x[3(n-3) + 1]
\end{split}
\end{equation}

Using this method of optimization the two more input and output port have been added becouse
now 3 inputs are processed and produce, at the same time, 3 output. To overall throughput has
been triplicate.

%% formula dove facciamo vedere che il throuput è triplicato e il nuovo valore calcolato

\section{Pipeline}

A further optimization has been applied. This method allows to reduce the size of critical path.

From schematic of previous FIR is possible to see that the critical path is the chain off 
adders. To achieve the best result is not necessary split each of them, but it is needed to 
group them in order to have a total time path that is less than the amount of time for performed
a multiplication. 

